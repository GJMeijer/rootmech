\documentclass[a4 paper, 11  pt]{article}

\usepackage{tabu}  %extended table
\usepackage{ctable}  %extended table
\usepackage{graphicx} %include figures 
\usepackage{epstopdf} %include eps files (transforms eps into pdf)
\usepackage[authoryear,round]{natbib} %style references
\usepackage{amsmath} %mathematical symbols
\usepackage{amssymb} %mathematical symbols 
%\usepackage{bm} %bold mathematical symbols
\usepackage{lineno} %line numbers
\usepackage[colorlinks=true, linkcolor=red, citecolor=blue]{hyperref} %format references - colored and hyperlinked
\usepackage{physics} %for differential d's
\usepackage{rotating}


%\usepackage{etoolbox}
%\AtBeginEnvironment{longtabu}{\footnotesize}{}{}

\textwidth=15.5cm
\textheight=24.0cm
\topmargin=-1cm
\oddsidemargin= 0.2cm
\evensidemargin= 0.2cm
\renewcommand{\topfraction}{0.95}
\renewcommand{\textfraction}{0.05}
\linespread{1.1} 	

\newcommand{\footremember}[2]{%
    \footnote{#2}
    \newcounter{#1}
    \setcounter{#1}{\value{footnote}}%
}
\newcommand{\footrecall}[1]{%
    \footnotemark[\value{#1}]%
} 
\title{Fitting root diameter counts with power law probabilities}
\author{%
  G. J. Meijer\footremember{uob}{Department of Architecture and Civil Engineering, University of Bath, Bath BA2 7AY, UK.}\footremember{corr}{Corresponding author, \texttt{gjm36@bath.ac.uk}}%
}
\date{\today}


\begin{document}
\maketitle
\linenumbers

\section{Segments}

The probability can be described using a multi-segmented power-law curves that is C$^0$ continuous. $n$ is the number of segments, separated by breakpoint values $x_b$ (length $n-1$). The minimum and maximum observed values of $x$ are given by $x_{min}$ and $x_{max}$, which automatically form the lower bound of the first segment and upper bound of the final segment when maximising the probability.

\section{Individual probabilities}

The probability in each segment $i$ is given by:
\begin{equation}
	p = \alpha_i x^{\beta_i}
\end{equation}
Because of C$^0$ continuity, at breakpoint location $j$:
\begin{equation}
	\alpha_{j} x_{b,j}^{\beta_{j}} = \alpha_{j+1} x_{b,j}^{\beta_{j+1}}
\end{equation}
so the multiplier $\alpha$ for the next segment satisfies:
\begin{equation}
	\alpha_{j+1} = \alpha_{j} x_{b,j}^{\beta_{j}-\beta_{j+1}}
\end{equation}

The probability in segment $j$ ($2 \leq j \leq n$) can now be described as:
\begin{equation}
	p_j = \alpha_1 x^{\beta_j} \prod_{i=2}^{j} x_{b,i-1}^{\beta_{i-1} - \beta_{i}}
\end{equation}
or alternatively as:
\begin{equation}
	p_j = \alpha_1 x^{\beta_1} \prod_{i=2}^{j} \left(\frac{x}{x_{b,i-1}}\right)^{\beta_{i} - \beta_{i-1}}
	\label{eq:pj}
\end{equation}

The log-transformed value of Equation \ref{eq:pj} is:
\begin{equation}
	\log\left(p_j\right) = \log\left(\alpha_1\right) + 
	\beta_1 \log\left(x\right) + 
	\sum_{i=2}^{j} \left(\beta_i - \beta_{i-1}\right)\left(\log\left(x\right) - \log\left(x_{b,i-1}\right)\right)
	\label{eq:logpj}
\end{equation}
Using the Heaviside function $H()$, the probability of any $x$ can be written as:
\begin{equation}
	\log\left(p\right) = \log\left(\alpha_1\right) + 
	\beta_1 \log\left(x\right) + 
	\sum_{i=1}^{n-1} \left(\beta_{i+1} - \beta_{i}\right)\left(\log\left(x\right) - \log\left(x_{b,i}\right)\right) H\left[\left(\log\left(x\right) - \log\left(x_{b,i}\right)\right)\right]
	\label{eq:logpj2}
\end{equation}
which is a function that can easily be vectorised in a numerical algorithm.

\section{Multiplication constant $\alpha_1$}

The unknown value of $\alpha_1$ should be set so that the total probability under the multi-segment curve (sum of total probabilities for each segment) is equal to one:
\begin{equation}
	1 = \sum_i \int \alpha_i x^{\beta_i} dx
\end{equation}
Therefore $\alpha_1$ can be expressed in terms of the $\beta$ and $x_b$ vectors. Define vector $\gamma$ as the vector containing all the boundaries of all segments:
\begin{equation}
	\gamma = [x_{min}, \vb{x}_b, x_{max}]
\end{equation}
The value of integral functions for each segment $i$:
\begin{equation}
	I_i = \int_{x=\gamma_i}^{\gamma_{i+1}} x^{\beta_i} dx
	=
	\begin{cases}
		\frac{\gamma_{i+1}^{\beta_i + 1} - \gamma_i^{\beta_i + 1}}{\beta_i + 1} &
		\text{when } \beta_i \neq -1 \\
		\log\left(\gamma_{i+1}\right) - \log\left(\gamma_i\right) &
		\text{when } \beta_i = -1
	\end{cases}
\end{equation}
We can write:
\begin{equation}
	1 = \sum_i \alpha_i I_i
\end{equation}
Given that:
\begin{equation}
	\frac{\alpha_i}{\alpha_1} = \prod_{j=1}^{i-1} x_{b,j}^{\beta_j - \beta_{j+1}}
\end{equation}
We can rewrite $\alpha_1$ as:
\begin{equation}
	\alpha_1^{-1} = \sum_{i=1}^{n} I_i \prod_{j=1}^{i-1} x_{b,j}^{\beta_j - \beta_{j+1}}
\end{equation}


\end{document}
